\documentclass{beamer}
\usepackage[utf8]{inputenc}
\definecolor{couleur}{rgb}{0.13, 0.67, 0.8}
\usecolortheme[named=couleur]{structure}
\usetheme{PaloAlto}
\usepackage{minted}

%%%%%%%%%%%%%%%%%%%%%%%%%%Page de garde%%%%%%%%%%%%%%%%%%%%%%%%%%
\title[] %optional
{Simulation du réseau cyclable de Montpellier}
\subtitle{HMMA238}

\author[]
{
\textsc{Alleau} Julie {\sc }  Master 1 Biostatistique \\  
\textsc{Chaoui} Wiam {\sc }   Master 1 Biostatistique\\  
\textsc{Poncheele} Clement {\sc }  Master 1 MIND     \\
\textsc{Seck} Gade {\sc }  Master 1 MIND \\}


\date[]  % (facultatif) 
{Année universitaire 2020-2021}

\logo{\includegraphics[height=1cm]{fds.png}}

%%%%%%%%%%%%%%%%%%%%%%%%%%Début du document%%%%%%%%%%%%%%%%%%%%%%%%%%
\begin{document}
\frame{\titlepage}

\begin{frame}
\frametitle{Table of Contents}
    \tableofcontents
\end{frame}


\section{Montpellier\_Biking}
\subsection{Introduction}

\begin{frame}
\frametitle{Introduction}
L'objectif final de ce projet est de créer un module Python capable de reproduire le flux des vidéos dans la ville de Montpellier pendant une semaine entière.
\end{frame}




\end{document}